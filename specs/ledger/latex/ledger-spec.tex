\documentclass[11pt,a4paper]{article}
\usepackage[margin=2.5cm]{geometry}
\usepackage{iohk}
\usepackage{microtype}
\usepackage{mathpazo} % nice fonts
\usepackage{amsmath}
\usepackage{amssymb}
\usepackage{latexsym}
\usepackage{mathtools}
\usepackage{stmaryrd}
\usepackage{extarrows}
\usepackage{slashed}
\usepackage[colon]{natbib}
\usepackage[unicode=true,pdftex,pdfa]{hyperref}
\usepackage{xcolor}
\usepackage[capitalise,noabbrev,nameinlink]{cleveref}
\usepackage{float}
\floatstyle{boxed}
\restylefloat{figure}
%%
%% Package `semantic` can be used for writing inference rules.
%%
\usepackage{semantic}
%% Setup for the semantic package
\setpremisesspace{20pt}

%%
%% Types
%%
\newcommand{\Bool}{\type{Bool}}
\newcommand{\Tx}{\type{Tx}}
\newcommand{\Ix}{\type{Ix}}
\newcommand{\TxId}{\type{TxId}}
\newcommand{\Addr}{\type{Addr}}
\newcommand{\UTxO}{\type{UTxO}}
\newcommand{\Value}{\type{Value}}
\newcommand{\Coin}{\type{Coin}}
\newcommand{\PrtclConsts}{\type{PrtclConsts}}
%% Adding witnesses
\newcommand{\TxIn}{\type{TxIn}}
\newcommand{\TxOut}{\type{TxOut}}
\newcommand{\VKey}{\type{VKey}}
\newcommand{\SKey}{\type{SKey}}
\newcommand{\Hash}{\type{Hash}}
\newcommand{\SkVk}{\type{SkVk}}
\newcommand{\Sig}{\type{Sig}}
\newcommand{\Data}{\type{Data}}
%% Adding delegation
\newcommand{\Epoch}{\type{Epoch}}
\newcommand{\VKeyGen}{\type{VKeyGen}}
%% Blockchain
\newcommand{\Gkeys}{\var{G_{keys}}}
\newcommand{\Block}{\type{Block}}
\newcommand{\CEEnv}{\type{CEEnv}}
\newcommand{\CEState}{\type{CEState}}
\newcommand{\BDEnv}{\type{BDEnv}}
\newcommand{\BDState}{\type{BDState}}
\newcommand{\Slot}{\type{Slot}}

%%
%% Functions
%%
\newcommand{\txins}[1]{\fun{txins}~ \var{#1}}
\newcommand{\txid}[1]{\fun{txid}~ \var{#1}}
\newcommand{\txouts}[1]{\fun{txouts}~ \var{#1}}
\newcommand{\values}[1]{\fun{values}~ #1}
\newcommand{\balance}[1]{\fun{balance}~ \var{#1}}
%% UTxO witnesses
\newcommand{\inputs}[1]{\fun{inputs}~ \var{#1}}
\newcommand{\wits}[1]{\fun{wits}~ \var{#1}}
\newcommand{\verify}[3]{\fun{verify} ~ #1 ~ #2 ~ #3}
\newcommand{\sign}[2]{\fun{sign} ~ #1 ~ #2}
\newcommand{\serialised}[1]{\llbracket \var{#1} \rrbracket}
\newcommand{\addr}[1]{\fun{addr}~ \var{#1}}
\newcommand{\hash}[1]{\fun{hash}~ \var{#1}}
\newcommand{\txbody}[1]{\fun{txbody}~ \var{#1}}
\newcommand{\txfee}[1]{\fun{txfee}~ \var{#1}}
\newcommand{\minfee}[2]{\fun{minfee}~ \var{#1}~ \var{#2}}
% wildcard parameter
\newcommand{\wcard}[0]{\underline{\phantom{a}}}
%% Adding ledgers...
\newcommand{\utxo}[1]{\fun{utxo}~ #1}
%% Delegation
\newcommand{\delegatesName}{\fun{delegates}}
\newcommand{\delegates}[3]{\delegatesName~#1~#2~#3}
\newcommand{\dwho}[1]{\fun{dwho}~\var{#1}}
\newcommand{\depoch}[1]{\fun{depoch}~\var{#1}}
%% Delegation witnesses
\newcommand{\dbody}[1]{\fun{dbody}~\var{#1}}
\newcommand{\dwit}[1]{\fun{dwit}~\var{#1}}
%% Blockchain
\newcommand{\bwit}[1]{\fun{bwit}~\var{#1}}
\newcommand{\bslot}[1]{\fun{bslot}~\var{#1}}
\newcommand{\bbody}[1]{\fun{bbody}~\var{#1}}
\newcommand{\bdlgs}[1]{\fun{bdlgs}~\var{#1}}

\begin{document}

%\input{frontmatter.tex}

\tableofcontents
\listoffigures

\section{Introduction}
\label{sec:introduction}
%\input{intro.tex}

\section{Notation}\label{sec:notation}

\begin{description}
\item[Powerset] Given a set $\type{X}$, $\powerset{\type{X}}$ is the set of all
  the subsets of $X$.
\item[Sequences] Given a set $\type{X}$, $\seqof{\type{X}}$ is the set of
  sequences having elements taken from $\type{X}$. The empty sequence is
  denoted by $\epsilon$, and given a sequence $\Lambda$, $\Lambda; \type{x}$ is
  the sequence that results from appending $\type{x} \in \type{X}$ to
  $\Lambda$.
\item[Functions] $A \to B$ denotes a \textbf{total function} from $A$ to $B$.
  Given a function $f$ we write $f~a$ for the application of $f$ to argument
  $a$.
\item[Fibre] Given a function $f: A \to B$ and $b\in B$, we write
  $f^{-1}~b$ for the \textbf{fibre} of $f$ at $b$, which is defined by
  $\{a \mid\ f a =  b\}$.
\item[Maps and partial functions] $A \mapsto B$ denotes a \textbf{partial
    function} from $A$ to $B$, which can be seen as a map (dictionary) with
  keys in $A$ and values in $B$. Given a map $m \in A \mapsto B$, notation
  $a \mapsto b \in m$ is equivalent to $m~ a = b$.
\end{description}

\section{Cryptographic primitives}
\label{sec:crypto-primitives}

Figure~\ref{fig:crypto-defs} introduces the cryptographic abstractions used in
this document.

\begin{figure}
  \emph{Abstract types}
  %
  \begin{equation*}
    \begin{array}{r@{~\in~}lr}
      \var{vk} & \SKey & \text{signing key}\\
      \var{vk} & \VKey & \text{verifying key}\\
      \var{hk} & \Hash & \text{hash of a key}\\
      \sigma & \Sig  & \text{signature}\\
      \var{d} & \Data  & \text{data}\\
    \end{array}
  \end{equation*}
  \emph{Derived types}
  \begin{equation*}
    \begin{array}{r@{~\in~}lr}
      (sk, vk) & \SkVk & \text{signing-verifying key pairs}
    \end{array}
  \end{equation*}
  \emph{Abstract functions}
  %
  \begin{equation*}
    \begin{array}{r@{~\in~}lr}
      \hash{} & \VKey \to \Hash
      & \text{hash function} \\
      %
      \fun{verify} & \VKey \times \Data \times \Sig
      & \text{verification relation}\\
    \end{array}
  \end{equation*}
  \emph{Constraints}
  \begin{align*}
    & \forall (sk, vk) \in \SkVk,~ m \in \Data,~ \sigma \in \Sig \cdot
      \verify{vk}{m}{\sigma} \iff \sign{sk}{m} = \sigma
  \end{align*}
  \emph{Notation for serialized and verified data}
  \begin{align*}
    & \serialised{x} & \text{serialised representation of } x\\
    & \mathcal{V}_{\var{vk}}{\serialised{m}}_{\sigma} = \verify{vk}{m}{\sigma}
      & \text{shorthand notation for } \fun{verify}
  \end{align*}
  \caption{Cryptographic definitions}
  \label{fig:crypto-defs}
\end{figure}

\section{Serialization}
\label{sec:serialization}

\begin{todo}
  Discuss here serialization and
  \href{https://iohk.myjetbrains.com/youtrack/issue/CDEC-628}{composable
    serialization}
\end{todo}

\section{UTxO}
\label{sec:state-trans-utxo-1}
%\input{utxo.tex}

\section{Delegation}
\label{sec:delegation}
\newcommand{\DCert}{\type{DCert}}
\newcommand{\DState}{\type{DState}}
\newcommand{\DEState}{\type{DEState}}

An agent owning a key that can sign new blocks can delegate its signing rights
to another key by means of \textit{delegation certificates}. These certificates
are included in the ledger, and therefore also included in the body of the
blocks in the blockchain.

In the blockchain protocol only a certain number of keys can sign blocks, and
the verifying part of these keys\footnote{Also known as public keys.} are
maintained in the genesis block. One important restriction on delegation is
that only the keys in the genesis block can delegate to other keys. However, at
the ledger level we do not know which are these keys, and thus this is a
restriction to be enforced at the blockchain level. In this formalization we
only care about establishing, whether $\var{vk}_s$ delegated its rights to
$\var{vk}_d$. To keep track of this, we use a map from keys to keys.

The rule for delegation is presented in
Figure~\ref{fig:rules:delegation}. It states that if $\var{c}$ is a valid
delegation certificate from key $\var{vk}_s$ to key $\var{vk}_d$, then the
delegation map $d$ is updated to contain the key mapping
$\var{vk}_s \mapsto \var{vk}_d$. The symbol $\unionoverride$ denotes
union-override, and is defined in Figure~\ref{fig:funcs:delegation}. Another
condition that is checked in the delegation rule is that each genesis key can
issue at most one delegation certificate per epoch. To check this, the states
of the transitions system maintain a map from epochs to set of keys that were
delegated on a given epoch. Finally, we only allow delegations for the next
epoch. This is what allow us to clean up the maps of keys delegated per epoch.
Variables to the left of the $\vdash$ (turnstile) symbol are the
\textit{environment} of a rule, which remains constant. This environment (the
current epoch in this case) would be set by the (chain) rules that use
Rule~\ref{eq:delegation}.

\begin{figure}
  \emph{Abstract types}
  %
  \begin{equation*}
    \begin{array}{r@{~\in~}lr}
      c & \DCert & \text{delegation certificate}\\
      vk_g & \VKeyGen & \text{genesis verification key}\\
      e & \Epoch & \text{epoch}\\
    \end{array}
  \end{equation*}
  \emph{Constraints}
  \begin{align*}
    \VKeyGen \subseteq \VKey
  \end{align*}
  \emph{Abstract functions}
  \begin{align*}
    & \fun{dwho} \in \DCert \mapsto (\VKeyGen \times \VKey) & \text{who delegates to who in the certificate}\\
    & \fun{depoch} \in \DCert \mapsto \Epoch & \text{certificate epoch}
  \end{align*}
  \caption{Delegation definitions}
  \label{fig:defs:delegation}
\end{figure}

\begin{figure}
  \begin{align*}
    & \unionoverride \in (A \mapsto B) \to (A \mapsto B) \to (A \mapsto B)
    & \text{union override}\\
    & d_0 \unionoverride d_1 = d_1 \cup (\dom d_1 \subtractdom d_0)
  \end{align*}
  \caption{Functions used in delegation rules}
  \label{fig:funcs:delegation}
\end{figure}

\begin{figure}
  \emph{Delegation states}
  \begin{align*}
    & \DState
      = \left(
        \begin{array}{r@{~\in~}lr}
          \var{dmap} & \VKeyGen \mapsto \VKey & \text{delegation map}
        \end{array}\right)
  \end{align*}
  \emph{Delegation transitions}
  \begin{equation*}
    \_ \vdash \_ \trans{deleg}{\_} \_ \in
      \powerset (\DState \times \DCert \times \DState)
    \end{equation*}
  \caption{Delegation transition-system types}
  \label{fig:ts-types:delegation}
\end{figure}

\begin{figure}
  \begin{equation}\label{eq:delegation}
    \inference[Delegation]
    {\dwho{c} = (vk_s, vk_d)
    }
    {
      \left(
      \begin{array}{r}
        \var{dmap}
      \end{array}
      \right)
      \trans{deleg}{c}
      \left(
      \begin{array}{lcl}
        \var{dmap} & \unionoverride & \{\var{vk_s} \mapsto \var{vk_d}\}
      \end{array}
      \right)
    }
  \end{equation}
  \caption{Delegation inference rules}
  \label{fig:rules:delegation}
\end{figure}

\begin{figure}
  \emph{Delegation transitions}
  \begin{equation*}
    \_ \vdash \_ \trans{deleg}{\_} \_ \in
      (\powerset \VKeyGen) \times \DCert \times (\powerset \VKey)
    \end{equation*}
  \caption{Delegation epoch transition-system types}
  \label{fig:ts-types:depoch}
\end{figure}

\begin{figure}
  \begin{equation}\label{eq:rule:depoch}
    \inference[Delegation]
    {\var{vk_s} \notin \var{ekeys}
    }
    {
      \left(
      \begin{array}{r}
        \var{ekeys}
      \end{array}
      \right)
      \trans{depoch}{c}
      \left(
      \begin{array}{lcl}
        \var{ekeys} \cup \{\var{vk_s}\}
      \end{array}
      \right)
    }
  \end{equation}
  \caption{Delegation epoch inference rules}
  \label{fig:rules:depoch}
\end{figure}


\subsection{Witnesses}
\label{sec:delegation-witnesses}

The rule for certificate witnesses is given in
Figure~\ref{fig:rules:delegationw}. The new definitions introduced in this rule
are given in Figure~\ref{fig:defs:delegationw}.

\begin{figure}
  \emph{Primitive types}
  \begin{equation*}
    \begin{array}{r@{~\in~}lr}
      \var{epoch} & \Epoch & \text{epoch}
    \end{array}
  \end{equation*}
  \emph{Abstract functions}
  \begin{equation*}
    \begin{array}{r@{~\in~}lr}
      \fun{dbody} & \DCert \to (\VKey \times \Epoch)
      & \text{body of the delegation certificate}\\
      \fun{dwit} & \DCert \to (\VKeyGen \times \Sig)
      & \text{witness for the delegation certificate}
    \end{array}
  \end{equation*}
  \caption{Delegation witnesses definitions}
  \label{fig:defs:delegationw}
\end{figure}

\begin{figure}
  \emph{Delegation with witness rule}
  \begin{equation}
    \label{eq:deleg-witnesses}
    \inference[Deleg-wit]
    { \dwit{c} = (\var{vk_s}, \sigma) & \verify{vk_s}{\serialised{\dbody{c}}}{\sigma} \\
      {\left(
        \begin{array}{r}
          \var{dmap}
        \end{array}
      \right)}
      \trans{deleg}{c}
      {\left(
      \begin{array}{l}
          \var{dmap}
      \end{array}
      \right)}
    }
    {
      {\left(
        \begin{array}{r}
          \var{dmap}
        \end{array}
      \right)}
      \trans{delegw}{c}
      {\left(
      \begin{array}{l}
          \var{dmap}
      \end{array}
      \right)}}
  \end{equation}
  \caption{Delegation witnesses inference rules}
  \label{fig:rules:delegationw}
\end{figure}


\section{Blockchain interface}
\label{sec:blockchain-interface}
\newcommand{\DIEnv}{\type{DIEnv}}
\newcommand{\DIState}{\type{DIState}}

\subsection{Delegation interface}
\label{sec:delegation-interface}

\begin{figure}
  \emph{Delegation interface environments}
  \begin{equation*}
    \DIEnv =
    \left(
      \begin{array}{r@{~\in~}lr}
        \var{e} & \Epoch & \text{epoch}\\
        \var{s} & \Slot & \text{slot}\\
        \var{d} & \Slot & \text{certificate liveness parameter}
      \end{array}
    \right)
  \end{equation*}

  \emph{Delegation interface states}
  \begin{equation*}
    \DIState
    = \left(
      \begin{array}{r@{~\in~}lr}
        \var{dmap} & \VKeyGen \mapsto \VKey & \text{delegation map}\\
        \var{sds} & \seqof{(\Slot \times (\VKeyGen \times \VKey))} & \text{scheduled delegations}\\
        \var{eks} & \powerset{\Epoch \times \VKeyGen} & \text{key-epoch delegations}
      \end{array}
    \right)
  \end{equation*}

  \emph{Delegation transitions}
  \begin{equation*}
    \_ \vdash \_ \trans{deleg}{\_} \_ \in
    \powerset (\DIEnv \times \DIState \times \seqof{\DCert} \times \DIState)
  \end{equation*}    
  \caption{Delegation interface transition-system types}
  \label{fig:ts-types:delegation-interface}
\end{figure}

\subsubsection{Delegation interface rules}
\label{sec:delegation-interface-rules}

\subsubsection{Delegation interface functions}
\label{sec:delegation-interface-functions}

TODO: define $\fun{delegates}$ and $\fun{initialDS}$.

% \section{Blockchain layer}
% \label{sec:blockchain-layer}
% \begin{note}
  This section provides a \textbf{proposal} on how the ledger rules can be used
  to build the blockchain ones. It was mainly developed to help me
  understanding what the blockchain layer requires from the ledger layer, and
  the aspects that need to be modeled in the former. In addition, my
  expectation with this section is that we can discuss which tasks should be
  completed in order to finish a first draft of the blockchain and ledger
  specifications, so that we can move forward with the generators. This section
  was not intended as a replacement of the blockchain spec, which can be found
  in a different document.-- Damian Nadales
\end{note}

\subsection{Chain extension}
\label{sec:chain-extension}

The chain extension rule is given in Figure~\ref{fig:rules:chain-extension},
and the definitions used in this rule are presented in
Figure~\ref{fig:defs:chain-extension}. Rule~\ref{eq:rule:chain-extension}
relies on transitions $\trans{bdeleg}{}$, which specify the delegation
behavior, and $\trans{butxo}{}$ which models the evolution of unspent outputs
after applying the transitions in a block. 

\begin{note}
  The protocol constants are currently modeled as a part of the
  chain extension environment. Once we adding voting
  (the mechanism which controls the protocol constants)
  to the model, it will probably make sense to move them to the
  chain extension states. The protocol constants will also be
  added to the ledger state, along with some kind of rule which
  combines voting with the UTxO rule:
  \begin{equation*}
    \inference[Voting-combine]
    {
      \var{pc} \trans{voting}{} \var{pc'} &
      \var{pc'} \vdash \var{utxo} \trans{utxo}{} \var{utxo'}\\ ~ \\
    }
    {\ldots}
  \end{equation*}
\end{note}

\begin{figure}
  \emph{Abstract types}
  \begin{equation*}
    \begin{array}{r@{~\in~}lr}
      \var{b} & \Block & \text{block}\\
      \var{s} & \Slot & \text{slot id}\\
    \end{array}
  \end{equation*}
  \emph{Abstract functions}
  \begin{equation*}
    \begin{array}{r@{~\in~}lr}
    \fun{bwit} & \Block \to (\VKey \times \Sig) & \text{block witness}\\
      \fun{bepoch} & \Block \to \Epoch & \text{block epoch}\\
      \fun{bslot} & \Block \to \Slot & \text{block slot id}\\
    \fun{s_0} & \Slot  & \text{slot zero (smallest slot id)}\\
    \end{array}
  \end{equation*}
  \caption{Blockchain extension definitions}
  \label{fig:defs:chain-extension}
\end{figure}

\begin{figure}
  \emph{Chain extension environments}
  \begin{equation*}
    \CEEnv =
    \left(
      \begin{array}{r@{~\in~}lr}
        \var{\Gkeys} & \powerset{\VKeyGen} & \text{genesis keys}\\
        \var{K} & \mathbb{N} & \text{number of nodes}\\
        \var{t} & \mathbb{Q} & \text{byzantine nodes ratio}\\
        \var{d} & \Epoch & \text{delegation liveness parameter}\\
        \var{pc} & \PrtclConsts & \text{protocol constants}\\
      \end{array}
    \right)
  \end{equation*}
  \emph{Chain extension states}
  \begin{equation*}
    \CEState =
    \left(
      \begin{array}{r@{~\in~}lr}
        \var{utxo} & \UTxO & \text{blockchain unspent outputs}\\
        \var{dmap} & \VKeyGen \mapsto \VKey & \text{blockchain delegation map}\\
        \var{signers} & \seqof{\VKeyGen} & \text{last $K$ blockchain signers}\\
        \var{sid_c} & \Slot & \text{current slot}\\
        \var{pdlgs} & \Slot \mapsto \seqof{\DCert} & \text{pending delegations}\\
        \var{ekeys} & \Epoch \times \powerset{\VKeyGen} & \text{keys delegated in the current epoch}
      \end{array}
    \right)
  \end{equation*}
  \emph{Chain extension transitions}
  \begin{equation*}
    \_ \vdash \_ \trans{chain}{\_} \_ \in
      \powerset (\CEEnv \times \CEState \times \Block \times \CEState)
  \end{equation*}
  \caption{Chain extension transition-system types}
  \label{fig:ts-types:chain-extension}
\end{figure}

\begin{figure}
  \begin{equation}
    \label{eq:rule:chain-base}
    \inference[Chain-base]
    {}
    {{\begin{array}{l}
         \Gkeys\\
         \wcard\\
         \wcard\\
         \wcard
      \end{array}}
      \vdash
      \left(
        \begin{array}{l}
          \var{utxo}\\
          \var{\{ (\var{vk}, \var{vk}) \mid \var{vk} \in \Gkeys\}}\\
          \emptyset\\
          \var{s_0}\\
          \emptyset\\
          \emptyset
        \end{array}
      \right)
    }
  \end{equation}

  \begin{equation}
    \label{eq:rule:chain-extension}
    \inference[Chain-ext]
    { \fun{bepoch}~b = \var{epo_c}
      & \bslot{b} = \var{sid_n} 
      & \var{sid_c} < \var{sid_n} \\
      {\begin{array}{l}
         \var{epo_c}\\
         \var{sid_n}\\
         d
       \end{array}}
      \vdash
      {
        \left(
          \begin{array}{l}
            \var{dmap}\\
            \var{pdlgs}\\
            \var{ekeys}
          \end{array}
        \right)
      }
      \trans{bdeleg}{b}
      {
        \left(
          \begin{array}{r}
            \var{dmap'}\\
            \var{pdlgs'}\\
            \var{ekeys'}
          \end{array}
        \right)
      }
      \\
      {\begin{array}{l}
         K\\
         t\\
         \var{sid_n}\\
         \var{dmap'}
      \end{array}}
      \vdash
      {
        \left(
          \begin{array}{l}
            \var{signers}
          \end{array}
        \right)
      }
      \trans{BSIGN}{b}
      {
        \left(
          \begin{array}{l}
            \var{signers'}
          \end{array}
        \right)
      }      
      &
      {
        \var{pc} \vdash
        \left(
          \begin{array}{l}
            \var{utxo}\\
          \end{array}
        \right)
      }
      \trans{butxo}{b}
      {
        \left(
          \begin{array}{r}
            \var{utxo'}\\
          \end{array}
        \right)
      }
    }
    {
      {\begin{array}{l}
         \Gkeys\\
         K\\
         t\\
         d\\
         pc
      \end{array}}
      \vdash
      {
        \left(
          \begin{array}{l}
            \var{utxo}\\
            \var{dmap}\\
            \var{signers}\\
            \var{sid_c}\\
            \var{pdlgs}\\
            \var{ekeys}
          \end{array}
        \right)
      }
      \trans{chain}{b}
      {
        \left(
          \begin{array}{l}
            \var{utxo}\\
            \var{dmap'}\\
            \var{signers'}\\
            \var{sid_n}\\
            \var{pdlgs'}\\
            \var{ekeys'}
          \end{array}
        \right)
      }
    }
  \end{equation}
  \caption{Chain extension rules}
  \label{fig:rules:chain-extension}
\end{figure}


\begin{figure}
  \begin{equation}
    \label{eq:rule:block-signing}
    \inference[Chain-ext]
    {\var{vk_g} \mapsto vk_d \in \var{dmap} & \bwit{b} = (\var{vk_d}, \sigma)\\
      \size{\{vk_g\} \restrictdom signers} \leq K * t &
      \verify{vk_d}{\serialised{\bbody{b}}}{\sigma} \\
      \var{signers'} =
         \{ \var{sid} \mapsto \var{vk}
          \mid  \var{sid} \mapsto \var{vk} \in \var{signers} \cup \{\var{sid_c} \mapsto vk_g\}
          , \var{sid_c} - K \leq \var{sid} \}
    }
    {
      {\begin{array}{l}
         K\\
         t\\
         \var{sid_c}\\
         \var{dmap}
      \end{array}}
      \vdash
      {
        \left(
          \begin{array}{l}
            \var{signers}
          \end{array}
        \right)
      }
      \trans{BSIGN}{b}
      {
        \left(
          \begin{array}{l}
            \var{signers'}
          \end{array}
        \right)
      }
    }
  \end{equation}
  \caption{Block signing rules}
  \label{fig:rules:block-signing}
\end{figure}


\addcontentsline{toc}{section}{References}
\bibliographystyle{plainnat}
\bibliography{references}

\end{document}
